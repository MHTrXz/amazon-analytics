\chapter{ساختار شکست کار}
در این فصل، ساختار شکست کار پروژه‌ی
\lr{Amazon Analytics}
نوشته شده است. این ساختار بر اساس فصل ششم اسلاید‌های استاد نوشته شده‌اند. در این فصل سعی شده است ساختار شکست کار، برای قسمت‌های \textit{کد نویسی} پروژه نوشته بشود، و به کارهایی که در گام اول و دوم پروژه مربوط هستند پرداخته نشده ‌است.

موارد تحویل دادنی که در فاز دوم تحت عنوان \textit{موارد تحویل دادنی} نوشته شده‌اند، نقطه سطح بالای شروع نوشتن ساختار شرکت کار برای پروژه 
\lr{Amazon Analytics}
هستند.

سپس این نقاط سطح بالا بر اساس ۳ قسمت اصلی فرض شده برای شرکت آمازون (\lr{Stock}، \lr{Site} و \lr{Shipment}) شکسته می‌شوند، و برای هر قسمت بسته‌های کاری کوچک‌تری نوشته می‌شوند.

یک سری از بسته‌های کاری نرم‌افزاری نبوده و نیاز به ماژول‌های سخت‌افزاری و نصب فیزیکی هستند، که از بسته‌ی کاری آنها صرفا یک اسم آورده شده و نقطه‌ی عطف آنها نصب و تست ماژول‌‌های سخت‌افزاری خواهد بود.


\section{وب سرویس جمع‌آوری داده}
وظیفه‌ی این وب سرویس
\begin{enumerate*}
\item 
جمع‌آوری داده،
\item 
ذخیره‌ی داده و
\item 
ارائه داده\RTLfootnote{ارائه داده یعنی از ابزارات قوی برای جستجو و برگرداندن داده از دیتابیس استفاده کند، برای مثال \lr{GraphQL}}
\end{enumerate*}
است.

این سرویس تمامی داده‌های مورد نیاز را از قسمت‌های مختلف و اصلی شرکت جمع‌آوری کرده، آنها به صورت بهینه ذخیره کرده و امکانات قدرتمند برای بازیابی اطلاعات را در اختیار استفاده کنندگان سرویس، قرار می‌دهد.

در ادامه راجع به قسمت‌های مختلف شرکت و ارتباط آنها با وب سرویس توضیح داده خواهد شد و در نهایت بسته شکست کاری این قسمت نشان داده می‌شود.

\subsection{\lr{Stock}}\label{ssec:stock}
قسمت \lr{Stock} تمامی کار‌‌های انبار‌داری شرکت را انجام می‌دهد. مهم‌ترین کار‌هایی که انجام می‌دهد عبارتند از
\begin{enumerate*}
\item 
وقتی کسی محصولات را می‌بیند، از موجود بودن یا نبودن‌ آن اطمینان حاصل کند،
\item 
اگر سفارشی ثبت شد، پردازش آن را انجام دهد،

\item 
محصول به حلقه‌ی بعدی زنجیره پردازش و تحویل کالا بدهد.
\end{enumerate*}

قسمت اعظمی از کار‌هایی که این قسمت انجام می‌دهد، به صورت سیستمی و کامپیوتری انجام می‌شوند؛ و نکته‌ی مهم اینست که برای \textit{رصد کردن} عملکرد این قسمت باید قسمت‌های نرم‌افزاری برای \lr{monitoring} به کد‌های این قسمت اضافه شوند. 

\subsection{\lr{Site}}
\subsection{\lr{Shipment}}

\subsection{ساختار شکست کار}
در این قسمت ساختار شکست کار‌ را نوشته‌ایم. دقت کنید که این جعبه‌ی بزرگ خود شامل ساختارهای شکست کار کوچک‌تر است که هر کدام نقطه‌ی عطف خودش را دارد.

\begin{wbsbox}{نوشتن وب سرویس}
\begin{wbssub}
{طراحی کلی وب ‌سرویس}
{نوشت شدن \lr{OpenAPI Specification} از روی موارد و تصمیم‌های بالا}
\task
بررسی و استخراج نیازمندی‌ها

\task
نوشتن نیازمندی‌ها

\task
استخراج کلی
\lr{endpoint}ها\RTLfootnote{یعنی بگوییم برای گرفتن داده‌های فلان قسمت \lr{stock} به یک \lr{endpoint} با این پارامتر‌ها نیاز داریم.}

\task
انتخاب نام برای \lr{endpoint}ها
\end{wbssub}

\begin{wbssub}
{پیاده‌سازی وب ‌سرویس}
{تست و دیپلوی شدن سرویس}
\task
بررسی معماری نرم‌افزاری وب سرویس

\task
تعیین و طراحی معماری

\task
انتخاب زبان برنامه‌نویسی

\task
انتخاب دیتابیس و \lr{toolchain} عه دیتابیس\RTLfootnote{یعنی انتخاب \lr{ORM} یا \lr{ODM}، و همچنین سیستم نگهداری و مدیریت \lr{migration}ها}

\task
پیاده‌سازی

\task
تست سرویس

\task
داکرایز کردن سرویس

\task
دیپلوی کردن سرویس
\end{wbssub}

\begin{wbssub}
{نوشتن کد‌های
\lr{monitoring}
برای موجود بودن یا نبودن محصول
(قسمت \ref{ssec:stock}\hspace{1mm}\lr{\nameref{ssec:stock}})}
{پاس شدن تمامی تست‌های لازم}
\task 
پیدا کردن توابع و کلاس‌هایی که اطلاعات لازم برای سرویس را تولید یا مصرف می‌کنند \\
\task 
اضافه کردن کد، برای ارسال اطلاعات به وب ‌سرویس \\
\task
تست کردن ارسال اطلاعات \\
\task
تست کردن دریافت صحیح اطلاعات
\end{wbssub}
\end{wbsbox}

\section{وب سرویس تحلیل داده و ارائه گزارش}

\section{وب اپلیکیشن}