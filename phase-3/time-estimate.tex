\chapter{تخمین زمان و تعداد نیروی کار}
در این فصل بر اساس نتایج بدست آمده در جدول
\ref{table:fps}
در فصل
\ref{chap:fps}،
به محاسبه‌ی نقاط تابعی می‌پردازیم و سپس بر اساس مدل کوکومو
\begin{itemize*}
\item 
تلاش،
\item 
مدت زمان و
\item 
تعداد افراد مورد نیاز 
\end{itemize*}
را محاسبه می‌کنیم.

\section{نقاط تابعی}
\subsection{نقاط تابعی تنظیم نشده \lr{(UAF)}}
\begin{table}[H]
\begin{center}
\caption{نقاط تابعی تنظیم نشده \lr{(UAF)}}
\begin{adjustbox}{width=\textwidth}
\begin{tabular}{rcccc}
\hline
& \multicolumn{4}{c}{پیچیدگی} \\
\cline{2-5}
&
پایین &
متوسط &
بالا &
جمع کل \\
\hline
فایل‌‌های منطقی داخلی \lr{(ILF)}&
$\theilfeasy \times 7 = 14$&
$\theilfnormal \times 10 = 220$&
$\theilfhard \times 15 = 150$&
384 \\
رابط خارجی \lr{(EIF)}&
$- \times 5 = -$&
$\theeifnormal \times 7 = 14$&
$- \times 10 = -$&
14 \\
ورودی خارجی \lr{(EI)}&
$\theeieasy \times 3 = 6$&
$\theeifnormal \times 4 = 8$&
$- \times 6 = -$&
14 \\
خروجی خارجی \lr{(EO)}&
$- \times 4 = -$&
$- \times 5 = -$&
$\theeohard \times 7 = 14$&
14 \\
استعلام خارجی \lr{(EQ)}&
$- \times 3 = -$&
$- \times 5 = -$&
$- \times 6 = -$&
0 \\
\textit{مجموع نقاط تابعی تنظیم نشده \lr{(UAF)}} &&&& 426 \\
\hline
\end{tabular}
\end{adjustbox}
\end{center}
\end{table}

\subsection{محاسبه‌ی فاکتور تطبیق مقدار}
طبق فرض این فاز، مقدار مجموع درجات تاثیر 
\lr{(TDI)}
برابر با ۵۰ است، پس مقدار 
\lr{VAF}
چنین می‌شود:
\begin{equation}\label{vaf}
\text{\lr{VAF}} = (50 \times 0.01) + 0.65 = 1.15
\end{equation}

\subsection{محاسبه‌ی نهایی نقاط تابعی}
با داشتن مقدار 
\lr{VAF} (\ref{vaf})
میتوان مجموع نقاط تابعی تنظیم شده را محاسبه نمود:
\begin{equation}
\text{\lr{FP}} = 426 \times 1.15 = 531.3
\end{equation}

\subsection{تعداد خط کد مورد نیاز}
تعداد خط کد مورد نیاز بر اساس این فرض که پروژه با زبان برنامه نویسی پایتون\RTLfootnote{\lr{Python (\url{python.org})}} نوشته می‌شود محاسبه شده‌ است.

\begin{equation}\label{loc}
\text{\lr{LOC (Python)}} = 48 \times 531.3 = 25502.4 \simeq 25503
\end{equation}

\section{مدل کوکومو}
پروژه‌ی 
\lr{Amazon Analytics}
یک پروژه \textit{تعبیه شده} تعریف می‌شود.

\subsection{تلاش \lr{(Effort)}}
فرمول محاسبه‌ی تلاش:
\begin{equation*}
\text{\lr{Effort}} = 3 \times \text{\lr{KDSI}}^{1.12}
\end{equation*}
با توجه به 
\ref{loc}
مقدار 
\lr{KDSI}
برابر با $25.503$ می‌باشد.
\begin{equation}\label{effort}
\text{\lr{Effort}} = 3 \times \underbrace{(25.503)^{1.12}}_{37.61} = 112.83
\end{equation}
\subsection{مدت زمان \lr{(Duration)}}
فرمول محاسبه‌ی مدت زمان:
\begin{equation*}
\text{\lr{Duration}} = 2.5 \times \text{\lr{Effort}}^{0.35}
\end{equation*}
و با توجه به 
\ref{effort}
مدت زمان برابر است با:
\begin{equation}\label{duration}
\text{\lr{Duration}} = 2.5 \times \underbrace{(112.83)^{0.35}}_{5.23} = 13.075 \ \ \text{ماه}
\end{equation}
\subsection{تعداد نفر مورد نیاز}
فرمول محاسبه‌ی تعداد نفر مورد نیاز
\begin{equation*}
\text{\lr{People}} = \frac{\text{\lr{Effort}}}{\text{\lr{Duration}}}
\end{equation*}
و با توجه به 
\ref{effort} و
\ref{duration}
تعداد افراد مورد نیاز برابر است با:
\begin{equation}
\text{\lr{People}} = \frac{112.83}{13.075} = 8.629 \simeq 8.63
\end{equation}