\section{وب اپلیکیشن}\label{sec:web-app}
می‌توان گفت چیزی که به عنوان 
\lr{Amazon Analytics}
به شرکت آمازون ارائه می‌شود، همین وب ‌اپلیکیشن است.\RTLfootnote{\lr{Web Application}}
در واقع 
\ref{sec:gatherer}
و
\ref{sec:processor}
دو قسمت 
\lr{internal}
این پلتفرم هستند و 
\lr{UI}یی
ندارند. چیزی که 
\lr{UI}
دارد و ورودی اصلی آن، تحلیل‌های 
\ref{sec:processor}
است، و آن‌ها را به روش‌های مختلفی همچون
\begin{enumerate*}
\item 
گزارشات کتبی،
\item
گزارشات آماری،
\item 
نمودار‌ها،
\item 
گراف‌‌ها و...
\end{enumerate*}
نمایش می‌دهد، همین وب‌ اپلیکیشن است.

\subsection{ساختار شکست کار}
\begin{wbsbox}{\nameref{sec:web-app}}
\begin{wbssub}
{تحلیل و تحقیق نیازمندی‌های استفاده کنندگان \lr{Amazon Analytics} برای \lr{UX}}
{سند نیازمندی‌ها}
\end{wbssub}

\begin{wbssub}
{طراحی فیگمایی \lr{UI} بر اساس تحلیل و تحقیق \lr{UX}}
{طراحی‌های فیگما}
\end{wbssub}

\begin{wbssub}
{نوشتن کد‌‌های \lr{UI} بر اساس طراحی‌های فیگما}
{کد‌های \lr{front-end}}
\end{wbssub}

\begin{wbssub}
{تست کردن کد‌های \lr{front-end}}
{پاس شدن تمامی تست‌های کد‌ها}
\end{wbssub}

\begin{wbssub}
{طراحی یک \lr{API} با سطح انتزاع بالاتر از \ref{sec:processor} برای گرفتن اطلاعات و \lr{decouple} کردن}
{نوشته شدن \lr{API}}
\end{wbssub}

\begin{wbssub}
{تست \lr{back-end}}
{پاس شدن تست‌ها}
\end{wbssub}

\begin{wbssub}
{وصل کردن کد‌های \lr{front-end} به \lr{back-end}}
{اجرای \lr{integration test} روی \lr{web app}}
\end{wbssub}
\end{wbsbox}
