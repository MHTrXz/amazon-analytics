\chapter{بخش ۲، تعریف پروژه\\\lr{(Project Definition)}}
\section{دورنما \lr{(Vision)}*}
ساخت یک سیستم خودکار، قدرتمند و دقیقِ ارائه دهنده‌ی گزارشات و تحلیل‌های دقیق از اطلاعات جمع‌آوری شده از قسمت‌های مختلف درخت سازمانی شرکت \lr{Amazon} به مدیران سطوح مختلف شرکت.

\section{اهداف \lr{(Objectives)}}
\subsection{اهداف تجاری*}
\begin{itemize}
\item 
تضمین صحت اطلاعات جمع‌آوری شده از قسمت‌های مختلف درخت سازمانی شرکت. (به علت خودکار بودن جمع‌آوری داده‌ها)
\item 
ارائه‌ی تحلیل‌های آماری-زمانی دقیق به مدیران ارشد و میانی شرکت، با استفاده از داده‌های قدیمی و جدید.

به علت در اختیار‌ داشتن داده‌های عملکردی جمع‌آوری شده از قسمت‌های مختلف، امکان تحلیل میزان رشد و بهبود در بهروری هر قسمت به مدیران فراهم می‌شود.
\item 
ارائه‌ی گزارشات عملکردی کارمندان و بخش‌های مختلف شرکت به مدیران سطوح میانی و رده‌بالای شرکت.\RTLfootnote{تفاوت گزارش و تحلیل در این است که گزارش صرفا نحوه‌ی نشان داده شدن یک سری داده را تعیین کرده و به گزارش‌گیر تحویل می‌دهد، اما در تحلیل سیستم با توجه به داده‌هایی که در اختیار دارد، آنها را تحلیل کرده و گزارشی کیفی و کمی از تحلیل خود به گزارش‌گیر ارائه می‌دهد.}
\end{itemize}
\subsection{اهداف فنی*}\label{tech-objective}
\begin{itemize}
\item
نوشتن یک وب‌سرویس برای جمع‌آوری و بایگانی داده‌های عملکردی کارکنان شرکت.
\item 
نوشتن یک وب‌سرویس برای تحلیل‌ و ارائه‌ی گزار‌ش‌های مورد نیاز به مدیران از اطلاعات جمع‌آوری شده توسط وب‌سرویس جمع‌آوری داده
\item 
نوشتن وب‌اپلیکیشن 
\lr{Amazon Analytics}
که با استفاده از دو وب‌سرویس بالا، از اطلاعات جمع‌آوری شده تحلیل و گزارش استخراج کند و به مدیران سطوح مختلف شرکت ارائه دهد.
\end{itemize}

\section{قلمرو \lr{(Scope)}}
\begin{itemize}
\item 
تمامی فرآیند‌های جمع‌آوری داده‌های عملکردی از دستی به خودکار تبدیل می‌شوند.

\item 
ارائه‌های گزارشات عملکردی به صورت دیجیتال و جزء به جزء انجام می‌شود.

\item 
فرآیند تحلیل داده‌های گذشته و ارائه‌های آمار‌های بهبود در قسمت‌های مختلف شرکت از دستی به خودکار عوض می‌شوند.

\item 
به دلیل وجود ساختاری سازمانی درختی و داده‌های کمی جمع‌آوری شده از این ساختار، مدیران می‌توانند به داده‌‌ها،‌ تحلیل‌ها و امکانات گزارش‌گیری تمامی زیر‌شاخته‌های خود (زیردستان مستقیم و غیر مستقیم) دسترسی داشته باشند.

\item 
تغییر در قسمت‌های از شرکت 
\lr{Amazon}
(به خصوص قسمت
\lr{Site}
و 
\lr{Shipment})
برای دریافت بازخورد از مشتریان.
\item 
تاثیر در عملکرد و تصمیم‌گیری‌های داخلی شرکت در مورد میزان بهروری کلی و قسمت‌های مختلف و کارمندان شرکت.

\item 
دیجیتالی سازی داده‌های جمع‌آوری شده از عملکرد کارکنان و بخش‌های شرکت.

\item 
ارائه دو وب‌سرویس و یک وب‌اپلیکیشن که در بخش
\nameref{tech-objective}
توضیح داده شدند.
\end{itemize}
\section{موارد تحویل دادنی \lr{(Deliverables)}}
\begin{table}[H]
\begin{adjustbox}{width=\textwidth}
\begin{tabular}{|m{2cm}|m{5cm}|m{6cm}|}
\hline
مورد &
اجزا &
شرح \\
\hline
\hline
وب‌سرویس جمع‌آوری داده &
\begin{itemize}
\item 
دریافت داده
\item
ذخیره‌ی داده
\item 
ارائه‌ی داده
\end{itemize}&

\begin{itemize}
\item
دریافت داده‌های عملکردی همه‌ی کارمندان و بخش‌ها
\item    
ذخیره‌ی صحیح و بهینه‌ی داده‌های دریافتی

\item
انعطاف در ارائه و
\lr{retrieve}
داده‌های ذخیره شده
\end{itemize}
\\
\hline
وب‌سرویس تحلیل داده و ارائه گزارش &
\begin{itemize}
\item 
 تحلیل داده
\item 
ارائه گزارش
\end{itemize}
&
\begin{itemize}
\item     
انجام تحلیل‌های مختلف روی داده‌های جمع‌آوری شده

\item
ارائه‌های گزارشات مختلف (گزارشات متنی، تصویر، نمودار و...) با سطوح دسترسی مختلف به مدیران
\end{itemize}

\\
\hline
وب اپلیکیشن &
\begin{itemize}
\item 
تجمیع دو سرویس گفته شده
\item 
بصری سازی
\item 
ارائه تنها رابط بین مدیران و \lr{Amazon Analytics}
\end{itemize}
& \\
\hline
\end{tabular}
\end{adjustbox}
\end{table}