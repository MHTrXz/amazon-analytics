\chapter{بخش ۵، ملاحظات پروژه\\\lr{(Project Considerations)}}\label{consd}

این فصل در مورد ریسک‌های پروژه 
\lr{Amazon Analytics}
بحث و بررسی مفصلی انجام می‌دهد.

لازم به ذکر است که ریسک‌های پروژه بر پایه‌ی ۳ ریسک اصلی نوشته شده‌اند:
\begin{enumerate}
\item 
بدست‌ آوردن داده،
\item 
اطمینان از دقیق و سالم بودن داده،
\item 
ارائه‌ی تحلیل دقیق.
\end{enumerate}

به علاوه فرض شده است که ریسک‌های مربوط به بودجه، تیم توسعه\RTLfootnote{برای مثال پیدا نکردن توسعه دهنده، رفتن یک توسعه‌دهنده و چنین ریسک‌های روتین} و چیز‌‌هایی از این قبیل، توسط خود شرکت آمازون فرض شده و استراتژی‌ها و پاسخ‌های مناسبی برای آنها در نظر گرفته شده، و در این فصل تنها به ریسک‌های مختص 
\lr{Amazon Analytics}
پرداخته شده است.

در بخش مفروضات، مفروضات لازم برای ریسک‌های ذکر شده در 
\ref{risks}،
گفته می‌شوند، اما ریسک‌هایی که خواهید خواند تماما بر اساس چارت سازمانی و تقسیم‌بندی انجام شده در سند مورد کاربرد تقسیم و نوشته شده‌اند.

\section{ریسک‌ها \lr{(Risks)}}\label{risks}

\subsection{ریسک‌های عملکردی}\label{op-risk}
\subsubsection{بخش \lr{Stock}}
\begin{itemize}
\item[\risk]
نداشتن اطلاعات دقیق از انبار‌های آمازون و پراکندگی کالاها
\item[\risk] 
نداشتن اطلاعات (خصوصا محل سکونت) خریداران یک کالا
\end{itemize}

\subsubsection{بخش \lr{Site}}
ریسک‌های این قسمت تاثیر زیادی از بازخورد‌های دریافتی از کاربران آمازون گرفته‌اند.

\begin{itemize}
\item[\risk]
کاربران به محصولات و قسمت‌‌های مختلف هیچگونه بازخوردی نمی‌دهند
\item[\risk]
فید‌بک‌های کاربران واقعی نیستند (یا با ربات تولید شده‌اند یا دروغ‌اند)
\end{itemize}

\subsubsection{بخش \lr{Shipment}}
\begin{itemize}
\item[\risk]
داده‌های ارسال شده توسط دستگاه‌های 
\lr{monitoring} 
صحیح نیستند
\item[\risk]
نداشتن اطلاعات و داده‌های مرحله‌ی تحویل
\item[\risk] 
نداشتن اطلاعات دقیق از توپوگرافی مناطقی که مامورین آمازون بسته‌ها را تحویل می‌دهند
\end{itemize}

\subsection{ریسک‌های زیرساختی}\label{inf-risk}
\subsubsection{بخش \lr{Data API}}
\begin{itemize}
\item[\risk]
نداشتن بک‌آپ از داده‌های مهم برای 
\lr{Amazon Analytics}

\item[\risk]
امن نبودن مکان‌های ذخیره‌ی دیتا (هم از لحاظ‌ امنیت اطلاعات هم از لحاظ حوادث فیزیکی)
\item[\risk]
مدیریت منابع سخت‌افزاری
\end{itemize}

\subsubsection{بخش \lr{Process API}}
\begin{itemize}
\item[\risk]
پایین بودن زمان پاسخ‌گویی 
\lr{Response time}

\item[\risk]
عدم وجود دقت لازم در فرمول‌ها و الگوریتم‌های تحلیل داده‌ها

\item[\risk]
مدیریت منابع سخت‌افزاری 
\end{itemize}

\subsubsection{بخش \lr{Web App}}
\begin{itemize}
\item[\risk]
امنیت دسترسی ضعیف باشد (برای مثال قسمت \lr{login})

\item[\risk]
رابط کاربری زیبا نباشد و تجربه‌ی کاربری سخت باشد

\item[\risk]
در دسترس نباشد

\item[\risk]
پاسخگرا (\lr{responsive}) نباشد

\item[\risk]
میزان تفکیک‌پذیری قابل قبول نباشد\RTLfootnote{این یعنی نتوان گزارش‌های گوناگون و البته جداجدا شده براساس رتبه‌ی گزارش‌گیر تحویل داد}

\item[\risk]
توانایی شاخه شاخه کردن و نشون دادن جزئیات هم به صورت کلی هم به صورت جزئی ضعیف باشد
\end{itemize}

\subsection{جدول ریسک‌ها}\label{risk-table}
جدول زیر بر اساس جدول واقع در اسلاید ۱۱.۸ استاد نوشته شده است.
\begin{table}[H]
\begin{center}
\begin{tabular}{|c|c|c|c|}
\hline
ریسک &
احتمال &
تاثیر &
امتیاز \\
\hline
\hline

۱ &
۴ &
۵۰\%&
$1.5$
\\
\hline
۲ &
۱ &
۳۰\%&
$0.3$\\
\hline
۳ &
۸ &
۴۰\%&
$3.2$\\
\hline
۴ &
۴ &
۵۰\%&
$2$\\
\hline
۵ &
۶ &
۶۰\% &
$3.6$\\
\hline
۶ &
۴ &
۴۰\%&
$1.6$\\
\hline
۷ &
۷ &
۴۰\%&
$2.8$\\
\hline
۸ &
۳ &
۷۰\%&
$2.1$\\
\hline
۹ &
۵ &
۳۰\%&
$1.5$\\
\hline
۱۱ &
۴ &
۳۰\%&
$1.2$\\
\hline
۱۲ &
۵ &
۶۰\%&
$3$\\
\hline
۱۴ &
۵ &
۸۰\%&
$4$\\
\hline
۱۵ &
۳ &
۴۰\%&
$1.2$\\
\hline
۱۶ &
۵ &
۷۰\%&
$3.5$\\
\hline
۱۷ &
۳ &
۲۰\%&
$0.6$\\
\hline
۱۸ &
۵ &
۷۰\%&
$3.5$\\
\hline
۱۹ &
۵&
۶۰\%&
$3$\\
\hline

۱۰ و ۱۳ &
۳&
۲۰\%&
$0.6$\\
\hline

\end{tabular}
\end{center}
\end{table}

\newpage
\subsection{رتبه‌بندی ریسک‌ها}\label{risks-rate}
\begin{enumerate}
\item 14
\item 5
\item 16
\item 18
\item 3
\item 12
\item 19
\item 7
\item 8
\item 4
\item 6
\item 1
\item 9
\item 11
\item 15
\item 10
\item 13
\item 17
\item 2
\end{enumerate}

\section{مشکلات \lr{(Issues)}}\label{issues}
این جدول الهام گرفته شده از جدول ۱۶.۸ اسلاید‌ها و نسخه موجود در 
\lr{project charter}
سامانه 
\lr{lms}
نوشته شده است.

به علت زیاد بودن ریسک‌ها، ۵ تا از ریسک‌‌ها با بالاترین اولویت توضیح داده شده‌اند.
\begin{table}[H]
\begin{center}
\begin{tabular}{|p{0.09\textwidth}|p{0.18\textwidth}|p{0.6\textwidth}|}
\hline
ریسک‌ها &
استراتژی &
پاسخ \\
\hline
\hline
14 &
اجتناب و تسکین &
از ابتدای شروع نوشتن پروژه، تست‌های امنیتی سختی باید برای آن نوشته و روی آن اجرا شوند. اگر مشکل امنیتی پیش آمد، باید \lr{endpoint} تحت تاثیر به صورت موقت از دسترس خارج شوند.\\
\hline
5 &
اجتناب و تسکین &
دستگاه‌های \lr{monitoring} ابتدا باید در شرایط سخت تست‌های لازم را پاس کنند. باید دستگاه‌های \lr{monitoring} جایگزین هم نصب شوند. اگر متوجه اطلاعات غلط ارسالی از دستگاه‌ها شدیم، داده‌های دریافتی را \lr{invalid} نشانه‌گذاری کرده و تا درست شدن آنها، چیزی را ذخیره نمیکنیم. \\
\hline
16 &
اجتناب &
باید نسخه‌های جایگزین برای این قسمت هم به صورت \lr{suspend} باشند تا در صورت \lr{down} شدن نسخه‌های دیگر بالا بیایند. باید از \lr{cloud service} خود آمازون \lr{(AWS)} برای مدیریت \lr{instance}‌های \lr{Amazon Analytics} استفاده کرد.\\
\hline
18 &
اجتناب &
باید قبل از دادن نسخه‌های پایدار برنامه، رابط کاربری برنامه را حتما با مدیران گزارش‌بگیر چک کردن و نیازمندی‌هایشان را برطرف ساخت. \\
\hline
3 &
 پذیرش و تسکین &
باید برای گرفتن بازخورد از روش‌های \lr{gamification} و روش‌‌های اقناع کننده استفاده کرد تا کاربران هر طور شده نظر بدهند.\\
\hline
\end{tabular}
\end{center}
\end{table}

\section{مفروضات \lr{(Assumptions)}}\label{assump}
در بررسی و مدیریت ریسک‌ها چنین مفروضاتی هست که
\begin{itemize}
\item
ساختار شرکت آمازون همانند شکل‌های سند فاز اول است.

\item
مقدمه‌ی سند \lr{Business Case} در نظر گرفته شده است.

\item 
ریسک‌های مربوط به مدیریت تیم و مباحث مالی توسط خود شرکت آمازون مدیریت می‌شوند.

\item 
زیرساخت‌های مورد نیاز توسط شرکت آمازون و پلتفرم \lr{AWS} تامین می‌شوند.
\end{itemize}

\section{محدودیت‌ها \lr{(Constraints)}}\label{const}
شرکت آمازون محدودیت مهمی از قبیل پول و زیرساخت برای این پروژه درنظر نگرفته است ولی از لحاظ زمانی، این پروژه باید طی دو سال آینده نوشته بشود.
