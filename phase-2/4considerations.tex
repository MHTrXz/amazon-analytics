\chapter{بخش ۵، ملاحظات پروژه\\\lr{(Project Considerations)}}

این فصل در مورد ریسک‌های پروژه 
\lr{Amazon Analytics}
بحث و بررسی مفصلی انجام می‌دهد.

لازم به ذکر است که ریسک‌های پروژه بر پایه‌ی ۳ ریسک اصلی نوشته شده‌اند:
\begin{enumerate}
\item 
بدست‌ آوردن داده،
\item 
اطمینان از دقیق و سالم بودن داده،
\item 
ارائه‌ی تحلیل دقیق.
\end{enumerate}

به علاوه فرض شده است که ریسک‌های مربوط به بودجه، تیم توسعه\RTLfootnote{برای مثال پیدا نکردن توسعه دهنده، رفتن یک توسعه‌دهنده و چنین ریسک‌های روتین} و چیز‌‌هایی از این قبیل، توسط خود شرکت آمازون فرض شده و استراتژی‌ها و پاسخ‌های مناسبی برای آنها در نظر گرفته شده، و در این فصل تنها به ریسک‌های مختص 
\lr{Amazon Analytics}
پرداخته شده است.

در بخش مفروضات، مفروضات لازم برای ریسک‌های ذکر شده در 
\ref{risks}،
گفته می‌شوند، اما ریسک‌هایی که خواهید خواند تماما بر اساس چارت سازمانی و تقسیم‌بندی انجام شده در سند مورد کاربرد تقسیم و نوشته شده‌اند.

\section{ریسک‌ها \lr{(Risks)}}\label{risks}

\subsection{ریسک‌های عملکردی}
\subsubsection{بخش \lr{Stock}}
\begin{itemize}
\item[\risk]
نداشتن اطلاعات دقیق از انبار‌های آمازون و پراکندگی کالاها
\item[\risk] 
نداشتن اطلاعات (خصوصا محل سکونت) خریداران یک کالا
\end{itemize}

\subsubsection{بخش \lr{Site}}
ریسک‌های این قسمت تاثیر زیادی از بازخورد‌های دریافتی از کاربران آمازون گرفته‌اند.

\begin{itemize}
\item[\risk]
کاربران به محصولات و قسمت‌‌های مختلف هیچگونه بازخوردی نمی‌دهند
\item[\risk]
فید‌بک‌های کاربران واقعی نیستند (یا با ربات تولید شده‌اند یا دروغ‌اند)
\end{itemize}

\subsubsection{بخش \lr{Shipment}}
\begin{itemize}
\item[\risk]
داده‌های ارسال شده توسط دستگاه‌های 
\lr{monitoring} 
صحیح نیستند
\item[\risk]
نداشتن اطلاعات و داده‌های مرحله‌ی تحویل
\item[\risk] 
نداشتن اطلاعات دقیق از توپوگرافی مناطقی که مامورین آمازون بسته‌ها را تحویل می‌دهند
\end{itemize}

\subsection{ریسک‌های زیرساختی}
\subsubsection{بخش \lr{Data API}}
\begin{itemize}
\item[\risk]
نداشتن بک‌آپ از داده‌های مهم برای 
\lr{Amazon Analytics}

\item[\risk]
امن نبودن مکان‌های ذخیره‌ی دیتا (هم از لحاظ‌ امنیت اطلاعات هم از لحاظ حوادث فیزیکی)
\item[\risk]
مدیریت منابع سخت‌افزاری
\end{itemize}

\subsubsection{بخش \lr{Process API}}
\begin{itemize}
\item[\risk]
پایین بودن زمان پاسخ‌گویی 
\lr{Response time}

\item[\risk]
عدم وجود دقت لازم در فرمول‌ها و الگوریتم‌های تحلیل داده‌ها

\item[\risk]
مدیریت منابع سخت‌افزاری 
\end{itemize}

\subsubsection{بخش \lr{Web App}}
\begin{itemize}
\item[\risk]
امنیت دسترسی ضعیف باشد (برای مثال قسمت \lr{login})

\item[\risk]
رابط کاربری زیبا نباشد و تجربه‌ی کاربری سخت باشد

\item[\risk]
در دسترس نباشد

\item[\risk]
پاسخگرا (\lr{responsive}) نباشد

\item[\risk]
میزان تفکیک‌پذیری قابل قبول نباشد\RTLfootnote{این یعنی نتوان گزارش‌های گوناگون و البته جداجدا شده براساس رتبه‌ی گزارش‌گیر تحویل داد}

\item[\risk]
توانایی شاخه شاخه کردن و نشون دادن جزئیات هم به صورت کلی هم به صورت جزئی ضعیف باشد
\end{itemize}

\begin{table}[H]
\begin{center}
\begin{tabular}{|c|c|c|c|}
\hline
ریسک &
احتمال &
تاثیر &
راهکار \\
\hline
\hline

۱ &
۴ &
۵۰\%&
\\
\hline
۲ &
۱ &
۳۰\%&
\\
\hline
۳ &
۸ &
۴۰\%&
\\
\hline
۴ &
۴ &
۵۰\%&
\\
\hline
۵ &
۶ &
۶۰\% &
\\
\hline
۶ &
۴ &
۴۰\%&
\\
\hline
۷ &
۷ &
۶۰\%&
\\
\hline
۸ &
۳ &
۷۰\%&
\\
\hline
۹ &
۵ &
۳۰\%&
\\
\hline
۱۱ &
۴ &
۳۰\%&
\\
\hline
۱۲ &
۵ &
۶۰\%&
\\
\hline
۱۴ &
۵ &
۸۰\%&
\\
\hline
۱۵ &
۳ &
۴۰\%&
\\
\hline
۱۶ &
۵ &
۷۰\%&
\\
\hline
۱۷ &
۳ &
۲۰\%&
\\
\hline
۱۸ &
۵ &
۷۰\%&
\\
\hline
۱۹ &
۵&
۶۰\%&
\\
\hline

۱۰ و ۱۳ &
۳&
۲۰\%&
\\
\hline

\end{tabular}
\end{center}
\end{table}
\section{مشکلات \lr{(Issues)}}

\section{مفروضات \lr{(Assumptions)}}

\section{محدودیت‌ها \lr{(Constraints)}}

